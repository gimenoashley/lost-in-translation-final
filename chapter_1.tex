%   Filename    : chapter_1.tex 
\chapter{Introduction}
\label{sec:researchdesc}    %labels help you reference sections of your document

\section{Overview}
\label{sec:overview}

Language is how humans communicate and express themselves \cite{Crystal_Robins_2024}. It evolves, adapting to the changing needs of users \cite{Jeresano_Carretero_2022}. New words are borrowed or invented \cite{article}, and most linguistic changes are initiated by young adults and adolescents (Thump, 2016 as cited in \cite{Jeresano_Carretero_2022}). The younger generation demographic tends to focus on belonging to self-organized groups of peers and friends, forming what can be described as the ``we" generation. Through their interactions, language changes differently, making them remarkably distinct from previous generations.

Slang is a great example of the dynamic nature of language. Slang is an informal language used by people in the same social group \cite{Fernández-Toro_2016}. It serves multiple social purposes: identifying group members, communicating informally, and opposing established authority \cite{McArthur_1998}. Slang is highly contextual and pervasive, even in non-standard English. Its figurative nature and how it twists the definitions of the words used make it difficult for outsiders to understand. 

In recent years, the Internet has become a significant medium for the evolution and spread of language, giving rise to `Internet slang' \cite{Liu_Zhang_Li_2023}. Internet slang is a collection of everyday language forms used by various online groups \cite{Barseghyan2014ONSA}. Ujang et al. (2018, as cited in \cite{Sabri2020}) state that internet slang is not easily understood by people outside the social group or people who are not fluent in the language where the slang is used. This phenomenon is particularly prominent among the younger generation \cite{Maulidiya_Wijaya_Mauren_Adha_Pandin_2021}, where they use it to communicate and interact with friends.

Generation Z, individuals born between 1996 and 2009, are regarded as ``digital natives" because technology is an integral part of their upbringing \cite{Dua_Jacobson_Ellingrud_Enomoto_Cordina_Coe_Finneman_2024}. Even the language of this generation is greatly affected by technology, where newly coined terms and phrases, called Gen Z slang, are tied to the media culture they've grown up with \cite{Jeresano_Carretero_2022}. However, this evolution of language often creates communication barriers with older generations (Venter, 2017 as cited in \cite{Ghazali_Abdullah_2021}). Furthermore, studies show that even within Generation Z, people with limited exposure to social media may struggle to understand the prevalent slang \cite{Vacalares_Salas_Babac_Cagalawan_Calimpong_2023}. 

These gaps highlight the need for a tool that can bridge the generational divide, making it easier for individuals to understand the language of Generation Z. Multiple studies have tried translating slang into a formal language using machine learning. Khazeni et al. achieved a 81.91\% accuracy in translating Persian slang to formal Persian language using deep learning. Another study by Nocon et al. created a translator to translate Filipino colloquialisms into the Filipino language using Tensorflow’s sequence-to-sequence model and Moses’ phrase-based statistical machine translation. Furthermore, Ibrahim and Sharief developed a slang translator using models from Hugging Face. 

Building on these studies, this study created a translation tool specifically to translate Gen Z slang. The tool will utilize Low Rank Adaptation (LoRA) to a selected Large Language Model (LLM). The results will be evaluated using the Recall-Oriented Understudy for Gisting Evaluation (ROUGE). 

By fostering mutual understanding, this tool aims to promote more effective and harmonious interactions across age groups, ultimately enhancing relationships and reducing miscommunication.

The main contributions of this study are as follows:
\begin{itemize}
	\item Enhance linguistic understanding between generations by using fine-tuning a LLM to translate Gen Z slang to formal language, leveraging the strengths of advanced NLP techniques
	\item Bridge communication gaps between generations using the proposed model to foster better relationships
	\item Create a scalable framework that can be adapted to translate slang in other languages
\end{itemize}

\section{Problem Statement}
\label{sec:problem_statement}

Internet slang fosters informal, relatable communication within the younger generation \cite{Ghazali_Abdullah_2021}, especially Generation Z, but it presents challenges in understanding for people outside this demographic. 
The gap in comprehension with older generations widens as internet slang evolves, often leading to miscommunication affecting social relationships that contribute to the generational divide \cite{Vacalares_Salas_Babac_Cagalawan_Calimpong_2023}. 
A more specific translation tool developed using language models can be used to bridge this divide.

By leveraging the ability of LLM to generate a more nuanced and properly constructed answer, a better tool can be made to translate the slang into proper sentences.
It has already been proven by the likes of GPT being modified and tailored for use in several automated chatbots to provide customer service.
However, no such tool exists for slang translation of Generation Z, which arguably has the most diverse slangs compared to other generations.
The creation of this tool will allow translating of such texts into formal sentences and help with bridging the generational divide between them and older people, especially teachers. 

%\subsection{Research Objectives} 
%\subsubsection{General Objective}
\section{Research Objectives}
\label{sec:research_objectives}

\subsection{General Objectives}
\label{sec:general_objectives}
This study aims to fine-tune the zephyr-7b LLM for use in the translation of Generation Z internet slang used by Filipinos in social media.
%\subsubsection{Specific Objectives}
\subsection{Specific Objectives}
\label{sec:specific_objectives}
Specifically, the study aims to:
\begin{enumerate}
	\item create a dataset of sentences containing Generation Z slang used in differing contexts and its formal translation,
	\item create a LoRA implementation for fine-tuning an existing model,
	\item fine-tune an existing LLM to translate sentences containing Generation Z slang into formal sentences, and
	\item evaluate the performance of the trained model and compare it to the baseline model using several performance metrics.
\end{enumerate}

%\subsection{Scope and Limitations of the Research} 
\section{Scope and Limitations of the Research}
\label{sec:scope}
This study focused on the use of internet slang by Filipino Generation Z, with an emphasis on the English language, as it is widely used across various digital platforms, including social media. English has become a dominant medium of communication in the Philippines' digital landscape, particularly among younger demographics. According to a study by \cite{Olobia_2024}, social media platforms serve as powerful tools for communicating in English as a second language, significantly influencing students' language use. The prevalence of English in social media facilitates learning opportunities and cross-cultural communication, highlighting its integral role in the digital communication practices of Filipino youth.

Furthermore, the extensive use of English on social media platforms reflects its status as a marker of education and social standing in the Philippines. As noted by Mateo (2018) cited by \cite{Esquivel_2020}, the widespread use of English in social media underscores its significance in Filipino society, where proficiency in English is often associated with educational attainment and social mobility.

These findings underscore the importance of focusing on English in studies of internet slang among Filipino Generation Z, as it remains a prevalent and influential language in their digital interactions.

%\subsection{Significance of the Research}
\section{Significance of the Research}
\label{sec:significance}
This study contributes to the growing body of research on the evolving linguistic landscape shaped by the use of Internet slang, highlighting the communication practices of Generation Z. As digital platforms become increasingly central to daily interactions, Generation Z continues to develop and adopt informal linguistic expressions that reflect their identity, creativity, and cultural environment. While this form of communication enhances peer connectivity, it can also create barriers for individuals outside this demographic, particularly older generations.

The findings of this study offer practical benefits for various stakeholders. For educators, the insights can support the development of more inclusive and responsive classroom communication strategies, enabling them to better understand and engage with their students' language use and cultural context. For parents, the study provides a framework for interpreting the language their children use online and in casual conversations, helping in bridging communication gaps and improving parent-child relationships. For media practitioners and digital marketers, understanding the patterns and meanings behind Gen Z slang can inform the creation of more relatable and culturally relevant content, enhancing audience engagement.

By addressing the communicative divide brought about by generational language differences, this research encourages a more informed approach to language variation in contemporary digital spaces. Ultimately, the study underscores the importance of adapting to linguistic change in order to foster clearer, more effective intergenerational communication.