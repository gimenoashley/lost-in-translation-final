%   Filename    : chapter_5.tex 
\chapter{Conclusion}
In this study, we constructed a dataset, containing 1,703 pairs of Gen Z internet slang sentences and their corresponding formal translations. We fine-tuned a zephyr-7B-Beta model and  evaluated its performance against the base model. Model training was stopped early to prevent overfitting, and the best model was selected based on validation performance. Both automatic and manual evaluation methods were employed to assess translation quality. Automatic metrics, using BLEU and ROUGE-L, showed that the fine-tuned model slightly outperformed the base model. Manual evaluation, conducted via online surveys with Generation Z students at UPV, indicated a moderate overall preference for the fine-tuned model, which received \textbf{53.5\% of the total vote}s. These results suggest that while the improvement in performance was not drastic, the fine-tuned model better aligned with the expectations and preferences of the target demographic.

\section{Limitations}
Language is dynamic and constantly evolving, making it difficult to establish clear boundaries on when slang terms emerge or fade within a generation. Additionally, the dataset created for this study was relatively small, and the number of evaluators involved was limited. In addition, as stated in Section 3.1.3, the computational constraints posed a challenge—loading a model with 7 billion parameters requires approximately 66 GB of memory, while Google Colab provided 16GB of VRAM which is insufficient for high-capacity models.

\section{Recommendations}
Future researchers are encouraged to expand the vocabulary of slang terms used on the Internet and explore more recent trends, taking into account the dynamic nature of language. It is also recommended that future studies utilize a larger and more diverse dataset to improve the robustness of the findings.