%   Filename    : abstract.tex 
\begin{center}
\textbf{Abstract}
\end{center}
\setlength{\parindent}{0pt}
Internet slang is an informal variation of language that is prominent to the younger generation. Its widespread use has contributed to the generational divide between younger and older generations. This study aimed to develop a translation tool leveraging Large Language Models (LLMs) to bridge this divide. A dataset of Generation Z slang sentences and their formal equivalents was used to fine-tune Zephyr-7B-Beta model. The performance of the fine-tuned model was evaluated against the base model using automatic metrics (BLEU and ROUGE-L) and manual evaluations through online surveys involving Gen Z students. The BLEU and ROUGE-L scores of 0.8151 and 0.8396 respectively, indicates a high degree of similarity between the generated text and the reference, suggesting that the model produces translations that closely match the formal equivalents of the Gen Z slang sentences. Furthermore, manual evaluation results showed that 53.5\% of the respondents preferred the translations produced by the fine-tuned model, supporting the results of the automatic metrics. The results suggest that fine-tuning LLMs can significantly improve their ability to translate internet slang into formal English. 

\begin{tabular}{lp{4.25in}}
\hspace{-0.5em}\textbf{Keywords:}\hspace{0.25em} & Internet Slang, Generation Z, Generational Divide, LoRA, LLM \\
\end{tabular}
