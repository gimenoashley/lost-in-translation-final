%   Filename    : abstract.tex 
\begin{center}
\textbf{Abstract}
\end{center}
\setlength{\parindent}{0pt}
Internet slang is an informal variation of language that is prominent to the younger generation. The usage of this language brought a generational divide between them and the older generations. This study aimed to develop a translation tool leveraging Large Language Models (LLMs) to bridge this issue. A dataset of Generation Z slang sentences and their formal equivalents was used to fine-tuned Zephyr-7B-Beta model. The performance of the fine-tuned model was evaluated against the base model using automatic metrics (BLEU and ROUGE-L) and manual evaluations through online surveys involving Gen Z students. Results showed that the fine-tuned model only slightly outperformed the base model in terms of automatic metrics, and it was generally preferred by human evaluators. These results indicate the fine-tuned model's effectiveness in producing more contextually appropriate and user-aligned formal translations.

\begin{tabular}{lp{4.25in}}
\hspace{-0.5em}\textbf{Keywords:}\hspace{0.25em} & Internet Slang, Generation Z, Generational Divide, LoRA, LLM \\
\end{tabular}
